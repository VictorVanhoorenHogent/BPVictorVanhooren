\chapter{\IfLanguageName{dutch}{Stand van zaken}{State of the art}}%
\label{ch:stand-van-zaken}

% Tip: Begin elk hoofdstuk met een paragraaf inleiding die beschrijft hoe
% dit hoofdstuk past binnen het geheel van de bachelorproef. Geef in het
% bijzonder aan wat de link is met het vorige en volgende hoofdstuk.

% Pas na deze inleidende paragraaf komt de eerste sectiehoofding.

In deze sectie van de paper wordt nagegaan wat voor onderzoek er reeds vericht is over het thesis onderwerp. Wanneer er gesproken wordt over het beveiligen van CI/CD omgevingen wordt er gebruik gemaakt van een nieuwe filosofie namelijk DevSecOps. In het eerste hoofdstuk van deze sectie wordt kort geschetst waar security thuis hoort binnen deze filosofie. Om te onderzoeken hoe secrets binnen een Jenkins pipeline het best beschermd worden, is het belangrijk eerst in kaart te brengen welke kwetsbaarheden bestaan binnen deze context. Het volgende hoofdstuk in deze sectie behandelt daarom bestaand onderzoek over deze kwetsbaarheden. In dit hoofdstuk wordt ook verder besproken welke aanvallen mogelijk zijn dankzij deze kwetsbaarheden en hoe deze aanvallen gebruikt kunnen worden om gevoelige informatie te extraheren binnen de pipeline. Het laatste hoofdstuk van deze sectie wordt gebruikt om samen te vatten wat voor onderzoek er reeds vericht is omtrent het beveiligen van de pipeline en meerbepaald het beveiligen van secrets voor een specifieke CI/CD oplossing.

\section{\IfLanguageName{dutch}{DevSecOps binnen de pipeline}{DevSecOps within the pipeline}}
\label{sec:DevSecOps binnen de pipeline}

\subsection{\IfLanguageName{dutch}{DevSecOps impact op de pipeline}{Devsecops impact on the pipeline}}
\label{sec:DevSecOps impact op de pipeline}
Om een beter beeld te schetsen van hoe secrets het best beveiligd worden, is het interessant de omgeving waarin deze gevoelige informatie gebruikt wordt te analyseren. De DevOps aanpak bestaat reeds lange tijd maar voldoet niet meer aan de security vereisten die vandaag gesteld worden. Tegenwoordig is er steeds meer een shift naar een andere filosofie, de DevSecOps filosofie. Om deze filosofie te adopteren is een strategie nodig. Een bedrijf kan niet van de ene op de andere dag DevSecOps implementeren. Deze filosifie zit verweven in de gehele software development life cycle (SDLC) \autocite{Jenkins}. \textcite{Ahmed2019} beschrijft in zijn master thesis welke invloed deze aanpak heeft op de SDLC. Het concept waar sofware pas getest wordt na het ontwikkelen is niet genoeg om veilige sofware af te leveren aan de klant. Er moet al vanaf de design fase aandacht gegeven worden aan security en security moet een rol spelen binnen elke fase van het development proces. Verder toont \textcite{Ahmed2019} aan met deze thesis dat wanneer deze aanpak correct geimplementeerd wordt, development teams alleen maar sneller betere sofware kunnen opleveren. Security is dus eerder een catalyst die het development proces naar een hoger niveau tilt.

\subsection{\IfLanguageName{dutch}{Het gebruik van tools binnen DevSecOps}{Using tools within DevSecOps}}
\label{sec:Het gebruik van tools binnen DevSecOps}
Om secrets te beschermen in een CI/CD omgeving is er nood aan verschillende tools die voor elke fase waar secrets gebruikt wordt bescherming bieden. Deze tools worden onderverdeeld in verschillende categorieen. In de systematische literatuur review van \textcite{Martelleur2022} wordt een overzicht gegeven van deze categorieen en de tools die het meest gebruikt worden binnen deze onderverdeling. Daarnaast onderzochten \textcite{Martelleur2022} ook waarom het gebruik van deze tools misschien eerder negatief kan zijn en wat de mogelijke uitdagingingen zijn wanneer deze tools gebruikt worden binnen een pipeline. Deze uitdagingen zijn niet enkel van toepassing op het gebruik van tools. Er zijn nog andere factoren die een impact hebben op de het gehele development team. \autocite{Rajapakse2022} spreekt over nog 3 andere domeinen die bepalen of een tool geadopteerd wordt op het einde van de rit. Er wordt gesproken over wrijving die kan ontstaan binnen de DevOpsWereld. Binnen de DevOps filosofie zijn er vaak gebruiken die clashen met gebruiken uit de security wereld. Ook de infrastructuur is een punt waar rekening mee gehouden moet worden. Sommige omgevingen zijn te complex of zijn voorzien van beperkte middelen om de overstap naar een andere filosofie te maken. De menselijke factor speelt ook een rol wanneer er gesproken wordt over DevSecOps. Niet iedereen beschikt over de nodige vaardigheden om security op de juiste manier te implementeren. Interne strubbelingen binnenin teams zijn een andere factor die vaak tot het falen leiden.

\subsection{\IfLanguageName{dutch}{DevSecOps tools om secrets te herkennen en beheren}{DevSecOps tools to recognize and manage secrets}}
\label{sec:DevSecOps tools om secrets te herkennen en beheren}
Hoe effecief deze hulpmiddelen zijn in het vinden van kwetsbaarheden in applicaties en bijgevolg secrets wordt door \textcite{Thulin2015} besproken in zijn master thesis. Vaak is er een combinatie van verschillende tools nodig om tot het beste resultaat te komen. Statisch analyse tools volstaan bij de meeste gevallen niet als security middel. Door het toevoegen van dynamische tools en analyses tijdens de applicatie runtime wordt de effectiviteit van deze tools verhoogd. Wanneer het over het detecteren van het fout gebruik van secrets gaat is het echter een ander verhaal. Het is zeer moeilijk om te detecteren wat als een secret beschouwd kan worden in een bepaalde context. Bijgevolg zijn er dus heel wat false positives wanneer gebruik wordt gemaakt van klassieke methodes. Met de komst van machine learning algoritmes is dit nadeel niet echt meer zichtbaar. Deze algoritmes zorgen ervoor dat de tool in kwestie leert herkennen wat een secret is. Op die manier is het dus veel efficienter om gevoelige informatie te herkennen.
\autocite{Saha2020}

\section{\IfLanguageName{dutch}{Kwetsbaarheden en aanvallen om secrets te extraheren}{Vulnerabilities and attacks to extract secrets}}%
\label{sec:Kwetsbaarheden en aanvallen om secrets te extraheren}

\subsection{\IfLanguageName{dutch}{Secret Sprawl}{Secret Sprawl}}
\label{sec:Secret Sprawl}
Er zijn heel wat studies uitgevoerd die duidelijk tonen dat er heel wat kwetsbaarheden zijn in de pipeline. \autocite{GitGuardian2021} ontwikkelde een tool om secrets te detecteren in Github repositories. In hun 2021 rapport wordt beschreven hoeveel secrets er nog altijd te vinden zijn online. Meestal is het niet de bedoeling dat de developer deze geheime informatie beschikbaar stelt. Wanneer er gewerkt wordt met een versiebeheer systeem echter, is het eenvoudig dit systeem fout te configuren. Ook worden alle secrets bewaart in de versiegeschiedenis van dit systeem en deze informatie publiekelijk beschikbaar. Doordat er steeds meer een tendens is naar het encrypteren van secrets en bijgevolg het minder toehankelijk maken van deze informatie, kan het gebeuren dat de developer kiest om deze gegevens te hardcoderen en op een onveilige manier te distribueren. Er zijn ook steeds meer secrets omdat er veel meer gebruik wordt gemaakt van cloudtechnologieen en er is gewoon meer informatie die beschermd moet worden. Een combinatie van deze factoren leidt tot een concept dat "Secret Sprawl" genoemd wordt.\autocite{GitGuardian2021}

\subsection{\IfLanguageName{dutch}{Bedrijgingen in de pipeline}{Threats in the pipeline}}
\label{sec:Bedrijgingen in de pipeline}
Waneer een aanvaller opzoek gaat naar mogelijke plaatsen waar hij de pipeline kan binnen dringen moet hij vaak niet ver zoeken. Secret Sprawl en het slordig gebruik van gevoelige informatie zorgt ervoor dat hij snel toegang krijgt tot de belangrijkste componenten van de omgeving. Op die manier kan de aanvaller de omgeving compromitteren in slechts enkele minuten. \autocite{Smart2022} Er is heel wat onderzoek vericht naar de verschillende bedrijgingen die voorkomen binnen de pipeline. \textcite{Rimba2015} beschrijft hoe er door het gebruik van security paterns een veiligere pipeline gecreeerd kan worden. In deze paper worden er ook een aantal risicos aangehaald die kunnen leiden tot het misbruik van de CI/CD omgeving. Secrets die publiekelijk beschikbaar zijn en als gebruiker toegang hebben tot de code repostiory zijn belangrijke bedrijgingen binnen deze omgeving. In ander onderzoek van \textcite{Ullah2017} worden er nog andere risico's aangehaald zoals het aanpassen van de pipeline met bedoeling toegang te krijgen tot de achterliggende infrastructuur. In dit onderzoek wordt vermeld dat voor elke component die deelneemt aan de pipeline evenveel aandacht aan beveiliging besteed moet worden. Authenticatie speelt een grote rol bij het toevoegen van extra beveilingslaag voor elke component. Om een gedetaileerdere lijst van bedrijgingen te verkrijgen kan er gebruik worden gemaakt van een threat model zoals STRIDE. \textcite{Paule2018} beschrijft in zijn master thesis aan de hand van het STRIDE model een aantal bedrijgingen en de kwetsbaarheden die hieraan gekoppelt kunnen worden. Door gebruik te maken van het STRIDE model is het mogelijk deze bedrijgingen te classificeren volgens het type aanval die veroorzaakt kan worden. In het onderzoek van \textcite{Paule2018} wordt het duidelijk dat niet enkel de componenten van de pipeline kwetsbaar zijn. De verschillende artefacten die geproduceerd worden doorheen de verschillende stadia zijn een andere doelwit die aanvallers kunnnen misbruiken. 

\subsection{\IfLanguageName{dutch}{Aanvallen in de pipeline}{Pipeline attacks}}
\label{sec:Aanvallen in de pipeline}
\textcite{Pecka2022} toont aan dat wanneer toegang verkregen wordt tot de CI/CD pipeline, het zeer eenvoudig is om geheime informatie op te halen voor bijvoorbeeld een Kubernetes cluster. Het is dan ook niet meer moeilijk om binnen te treden in de achterliggende infrastructuur. Niet alleen de pipeline is een doelwit voor aanvallers alle componenten die deelnemen in de gehele cyclus zijn intressanten doelwitten. \textcite{Security2022} beschrijven in 1 van hun risico's hoe het mogelijk is omgevingsvariabelen uit te lezen via een “Poisened Pipeline Execution” aanval. Doordat variabelen ingesteld worden op “configuration as code” niveau, is de kans op dergelijke aanvallen groot. Daarnaast geven ze ook aandacht voor aanvallen waarbij de onderdelen van de omgeving misbruikt worden. Het misbruiken van dependencies waarop bepaalde onderdelen in de pipeline steunen en het misbruiken van plugins die gebruikt worden binnen de pipeline scripts zijn slechts enkele aanvallen die tot deze categorie behoren. Hygiënisch omgaan met secrets en vertrouwelijke informatie blijft een belangrijk aspect om deze aanvallen tegen te gaan. \autocite{Haymore2022} omschrijven in een artikel van NCC Group verschillende aanvallen die mogelijk zijn wanneer op een incorrecte manier gebruik gemaakt wordt van een specifieke technologie. Een Amazon S3 bucket wordt gebruikt om gegevens op te slaan, maar als deze fout geconfigureerd is, is dit het perfect toegangspunt voor de aanvaller. Sommige plugins die beschikbaar voor Jenkins zijn vaak niet duidelijk omschreven. Wanneer deze plugins gebruikt worden in een omgeving en deze zijn dus fout geconfigureerd, leidt dit tot een ander toegangspunt voor de aanvaller.

\subsection{\IfLanguageName{dutch}{Praktische voorbeelden uit de security wereld}{Practical examples from the security scene}}
\label{sec:Praktische voorbeelden uit de security wereld}
Het CI/CD framwork GitLab maakt gebruik van runners voor het uitvoeren van CI/CD jobs. \textcite{Haymore2022} beschrijven hoe GitLab misbruikt kan worden als deze omgeving op een foute manier geconfigureert is. De meeste aanvallen die mogelijk zijn, ontstaan door componenten in GitLab die te veel permissies hebben. Bijvoorbeeld docker containers die uitgevoerd worden met verkeerde rechten waardoor de aanvaller uit deze omgeving kan ontsnappen. GitLab runners die secrets blootleggen aan CI/CD jobs die niet genoeg beveiligd zijn en repository branches met te veel rechten zijn voorbeelden van hoe groot de attack surface kan worden. Op het evenement Black Hat USA 2022 \footnote{https://www.blackhat.com/us-22/} verklaarden \textcite{Smart2022} waarom het van cruciaal belang is dat er op een correcte manier omgegaan wordt met secrets. Bij hun aanvallen op de CI/CD pipelines, komen zij vaak het eerst in aanraking met secrets en daardoor is het zeer eenvoudig toegang te verkrijgen tot de achterliggende software. Verschillende aanvallen zijn niet alleen te danken aan het slordig gebruik van secrets, maar ook het niet toepassen van least privilege Roll Based Access Control (RBAC). Door zich voor te doen als legitieme bron en door misbruik te maken van incorrecte configuraties is het kinderspel binnen te breken in een CI/CD omgeving. De soort pipeline die gebruikt wordt is vaak de oorzaak van het probleem. De gebruiker beschikt niet over de nodige kennis om deze omgeving te beveiligen. Uit een onderzoek van \textcite{Koishybayev2022} wordt het duidelijk dat bepaalde CI/CD software gewoon zelf niet over voldoende beschermingsmaatregelen beschikken om een veligie pipeline te kunnen garanderen. Geheime informatie kan lekken doordat deze in de logs beschikbaar is, maar ook doordat de secrets beschikbaar zijn voor het gehele systeem. Zij halen aan dat enkel de plugins en stappen waar de secrets gebruikt worden toegang mogen hebben tot deze informatie. De sofware die dagelijks gebruikt wordt binnen deze omgevingen is ook een kwetsbaar punt. Zeer recent was er een kwetsbaarheid in TerraForm waardoor Remote Code Execution mogelijk was.\autocite{Suezawa2021}\autocite{Kaskasoli2021}\autocite{Frank2021} spreken allemaal over de aanval die mogelijk was wanneer gebruik werd gemaakt van Terraform. Op DEFCON 29 \footnote{https://defcon.org/html/defcon-29/dc-29-index.html} toont \textcite{Ahmed2021} aan dat er nog tal van aanvallen mogelijk zijn binnen de Terraform omgeving. De Terrform API bijvoorbeeld is kwetsbaar omdat er gevoelige informatie opgehaald kan worden. Providers waar gebruik van wordt gemaakt binnen de build scripts in de Terraform omgeving kunnen gecompromiteerd zijn en op die manier kan een achterpoort ingebouwd worden in de pipeline.Infrastructure as code (IAC) is een andere punt waarlangs een aanval mogelijk is. \textcite{Suezawa} beschrijft in zijn presentatie voor CODE BLUE 2021 OpenTalks \footnote{https://codeblue.jp/2021/en/} hoe een hacker de gehele cloud infrastrctuur kan binnendringen omdat de IAC omgeving niet correct geconfigureerd is. In DEFCON 29 toont \textcite{Bar2021} aan dat zelf de statisch tools die gebruikt worden om de security van applicaties te testen kwetsbaarheden bevatten. Veel van deze kwetsbaarheden leiden tot het uitvoeren van code op het syteem, maar sommige kunnen ook andere aanvallen veroorzaken zoals denial of service attacks. Vooraleer tools gebruikt worden in het development proces is het belangrijk dat het security team nadenkt hoe deze tools misbruikt kunnen worden. 
% mss nog toevoegen wat er wordt gedaan met deze credentials

\section{\IfLanguageName{dutch}{Beveiligingsmaatregelen om secrets te beschermen}{Security measures to protect secrets}}%
\label{sec:Beveiligingsmaatregelen om secrets te beschermen}

\subsection{\IfLanguageName{dutch}{Secrets in versie beheer en basis security principes}{Secrets in version control and basic security principles}}%
\label{sec:Secrets in versie beheer en basis security principes}
De eerste stap om deze aanvallen te verhelpen ligt bij het volgen van de basis security principes, \autocite{Smart2022}, \autocite{Haymore2022}, \autocite{Suezawa2021} halen deze principes in hun onderzoek steeds terug aan. Least privilege (RBAC) toepassen in de omgeving, zoveel mogelijk de verschillende componenten isoleren wanneer mogelijk, security toepassen in verschillende lagen en security monitoring toepassen zijn belangrijke principes die zeker moeten toegepast worden. Vooraleer de secrets gebruikt zullen worden voor het bouwen van infrastructuur of voor het bouwen van applicaties, is het belangrijk dat de versie beheer systemen (VCS) van de nodige veiligheids maatregelen voorzien zijn. Indien de VCS niet op een correcte manier beveiligd wordt duurt het niet lang vooraleer de threat actor toegang heeft.\autocite{Mouw2021}.In een grijze literatuurstudie van \textcite{Basak2023} worden enkele best practices besproken omtrent het gebruik van secrets in VCS. Het beste is om de secrets zoveel mogelijk weg te laten uit de repositories. Als dit geen optie is, kan er gekozen worden om deze informatie te bewaren in configuratie bestanden. Het gebruik van gitignore is zeer belangrijk binnen een VCS. Private repositories zijn niet veilig en worden best niet gebruikt om secrets te bewaren. Zorg ervoor dat wanneer er gecommit wordt naar de VCS dat er geen secrets geleaked kunnen worden. Tools zoals TruffleHog, Gitrob en git-all-secrets kunnen helpen met het scannen van de VCS. Door gebruik te maken van git hooks en git flags bij de pre commit en post commit fase is het ook mogelijk commits tegen te houden. Indien secrets toch niet tegen gehouden worden door deze maatregelen, is het belangrijk deze credentials in te trekken en de VCS geschiedenis te verwijderen. Er mag vanuit gegaan worden dat wanneer deze situatie zich voordoet dat deze geheimen gecompromiteerd zijn vanaf dat moment. De beheerders van deze sofware kunnen gecontacteerd worden om de secrets te verwijderen uit de cache in sommige gevallen. Het blijft belangrijk de basis security principes in acht te nemen.

\subsection{\IfLanguageName{dutch}{Secrets in infrastructure as code en secrets beheer}{Secrets in infrastructure as code en secrets management}}%
\label{sec:Secrets in infrastructure as code en secrets beheer}
CI/CD pipelines worden niet enkel gebruikt om applicaties te bouwen, een groot deel van deze pipelines is gerserveerd voor het opbouwen van infrastructuur. Deze infrastructuur wordt opgebouwd aan de hand van code en is een andere component van de CI/CD omgeving die beschermd moet worden. Veel van deze scripts bevatten nog te veel hard-code secrets. Gebruikers maken niet op de juiste manier gebruik van de ingebouwde tools die beschikbaar zijn om de secrets te beschermen.\autocite{Rahman2019},\autocite{Kumara2021} \textcite{Morris2021} bespreekt in zijn boek verschillende manieren waarop secrets beschermd kunnen worden binne de IAC wereld. Encryptie van secrets, authorisatie zonder secrets, injectie van secrets tijdens runtime en wegwerpbare secrets zijn enkele initiatieven die genomen kunnen worden om een veiligere IAC omgeving te creeeren. In het onderzoek van \textcite{Rahman2021} wordt uitgebreider ingegaan op verschillende van deze categorieen en worden ook nog enkele andere belangrijke punten besproken. Encrypteer niet zomaar alle gegevens, ga na welke belangrijk zijn en prioriteit krijgen. Wanneer authorisatie zonder secrets geen mogelijkheid is implementeer het juiste access control beleid. Het basis principe RBAC is een van de beste manieren om access control aan te pakken. Het gescheiden houden van artifacten, het loggen van alle operaties die gebeuren met secrets en het roteren van secrets zijn andere belangrijke aandachtspunten wanneer er gewerkt wordt met deze geheime informatie. Zoals al eerder aangehaald werd in een vorige paragraaf zijn scanning tools een extra maatregel die kan helpen in de strijd tegen de cybercriminelen. CredScan en SLIC zijn voorbeelden van scanners voor de IAC omgeving. In de ideale situatie is er naast deze maatregelen ook een kluis voorzien voor de verschillende secrets. Binnenin deze kluis zijn heel wat van de opgesomde maatregelen ingebouwd en is het beheer van alle secrets gemakkelijker. Deze tools zorgen er ook voor dat de voetafdruk van de secret niet zichtbaar is in de logs of shell geschiedenis.\autocite{Agarwal2021} Verschillende onderzoekers hebben gedocumenteerd op welke manier er het best omgegaan wordt met secrets. In de paper van \autocite{Basak2023} wordt advies om secrets te beschermen beschreven. Heel wat van het advies is gebaseerd op dezelfde punten die aangehaald worden wanneer met IAC scripts gewerkt wordt, maar er is ook heel wat nieuw advies. Het gebruik van environment variabelen tijdens runtime, het inladen van secrets via een extern systeem zijn verschillende manieren om deze geheime informatie te beschermen. Secrets in de pipeline ondervinden verschillende fasen. In alles fasen waar secrets zich bevinden is het belangrijk maatregelen toe te passen. Secrets in rust kunnen geencrypteerd worden en beschermd worden door middel van secret management sofware. Secrets in de gebruiksfase worden ook beste beschermd op dezelfde manier. Daarnaast door te steunen op het basis principe van het scheiden van omgevingen wordt deze secrets extra beschermd.\autocite{Basak2022} In het  boek \textcite{Calles2020} wordt het secrets verhaal meer vanuit een business perspectief bekeken. De best practices besproken in dit boek leunen sterk aan bij de voorgaande, maar er wordt ook besproken hoe de cloud gebruikt kan worden om secrets te managen. Amazon, Google, Microsoft hebben allemaal een key management systeem, waardoor secrets in de cloud op een veilige manier gebruikt kunnen worden.

% boeken paginas en hoofdstuk vermelden
\subsection{\IfLanguageName{dutch}{Beveiligen van de pipeline}{Securing the pipeline}}%
\label{sec:Beveiligen van de pipeline}
Wanneer het komt tot het beveiligen van de pipeline zijn er heel wat plaatsen waar secrets gebruikt worden. Containers worden steeds meer gebruikt en ook in pipelines. Het beveiligen van containers is daarom zeer belangrijk. Risico's zoals secrets die ingebakken zitten in de image kunnen opgemerkt worden met container scanners. Wanneer deze tools secrets of andere kwetsbaarheden detecteren is er de mogelijkheid om het bouwproces te laten falen zodat mogelijkse aanvallen vermeden kunnen worden.\autocite{Agarwal2021} \textcite{Security2022} beschrijven naast verschillende aanvallen ook mitigatie technieken en maatregelen die genomen kunnen worden om aanvallen te beperken. Heel wat van deze maatregelen komen neer op de basis principes van security en kunnen ook toegepast worden om verschillende aanvallen in te perken.\textcite{Suezawa2021} beschrijft in zijn presentatie ook enkele maatregelen die uitgevoerd kunnen worden op pipeline niveau. De meeste van deze maatregelen zijn gedocumenteerd in zijn threat matrix. Daarnaast is de ontwikkeling van beveiligingsmatrices voor de CI/CD omgeving iets waar naar gewerkt wordt.  

\subsection{\IfLanguageName{dutch}{Security framework en bedrijgingsmatrix}{Security framework and threat matrix}}
\label{sec:Security framework en bedrijgingsmatrix}
Wanneer er over het beschermen van CI/CD omgevingen gesproken wordt, is er heel wat advies te vinden in de wetenschappelijke werken die voorheen vermeld zijn. Het is echter moeilijk om een plan op te stellen aan de hand van deze informatie. \autocite{Koopman2019AFF} beschrijven een andere aanpak door het opstellen van een framework dat gebruikt kan worden voor het detecteren van kwetsbaarheden en bedrijgingen. Op die manier is het mogelijk een strategie uit te werken die gebruikt kan worden binnen de organisatie. Recent is er een document ontwikkeld waarin duidelijk de 10 meest voorkomende risico's voor de pipeline voorkomen. \textcite{Security2022} in samenwerking met OWASP beschrijft in dit document wat het risico's precies is wat de impact is van dit risico en welke maatregelen er genomen kunnen worden om de schade te beperken. Door dit document te maken in samenwerking met OWASP is er een bepaald classificatiesysteem die gebruikt kan worden om deze risco's te herkennen. Heel wat van deze risco's in dit document zijn ondersteund van referenties naar aanvallen waarbij er gebruik is gemaakt van deze risico's. Gelijkaardig aan de MITRE Att\&CK\textregistered matrix, heeft Mercari Security Team een threat matrix \footnote{https://github.com/rung/threat-matrix-cicd} ontwikkeld die gebruik kan worden binnne CI/CD omgevingen. Met behulp van deze matrix en een framework is het veel realitischer om een plan van aanpak op te stellen voor het beveiligingen van de CI/CD omgeving en bijgevolg het voorkomen van een supply chain aanval. \textcite{Foundation2021} beschrijven in hun document de hoofd principes die best gevolgd worden om de veiligheid van de supply chain te garanderen. Via dit document is er een duidelijk stappenplan om de pipline te beveiligen, maar er wordt nog heel wat verder gegaan want alle subcomponenten die gebruikt worden in de pipeline of die op een of de andere manier een bepaalde invloed uitoefenen op de pipeline zijn mogelijke punten waar de aanvaller misbruik kan van gaan maken. Bepaalde organisaties zijn bezig met het maken van stadaarden en controles om uiteindelijk een beter security ecosysteem te creeeren. SLSA \footnote {https://slsa.dev} is een voorbeeld van een framework die in ontwikkeling is met dit doeleinde. Door van standaarden gebruik te maken is het gehele security verhaal veel vatbaarder geworden niet alleen voor bedrijven. 


